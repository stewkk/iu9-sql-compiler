\documentclass{beamer}
\usetheme{metropolis}           % Use metropolis theme

\usepackage[T2A]{fontenc}
\usepackage[utf8]{inputenc}
\usepackage[english, main=russian]{babel}

\usepackage{listingsutf8}
\usepackage[ruled]{algorithm}
\usepackage{algorithmicx}
\usepackage[noend]{algpseudocode}
\usepackage{amsmath}
\usepackage{caption}
\usepackage{graphicx}
\usepackage{subcaption}
\usepackage{float}

\usepackage{listings}

\lstdefinestyle{mystyle}{
    basicstyle=\ttfamily\footnotesize,
    breakatwhitespace=false,
    breaklines=true,
    captionpos=b,
    keepspaces=true,
    showspaces=false,
    showstringspaces=false,
    showtabs=false,
    tabsize=2
}

\lstset{style=mystyle}

\usepackage{tikz}
\usetikzlibrary{shapes.misc}
\usetikzlibrary{trees}

\floatname{algorithm}{Листинг}
\renewcommand{\algorithmname}{Листинг}

\title{Оптимизирующий компилятор запросов подмножества SQL на основе LLVM}
\date{\today} \author{Старовойтов Александр} \author[me]{Выполнил: Старовойтов Александр
  ИУ9-71Б\\[1mm]Руководитель: Непейвода А.Н.}
% \thanks
% \institute{МГТУ им. Н.Э. Баумана}
\begin{document}
\maketitle

\begin{frame}{Мотивация}
  \begin{alertblock}{}
    \begin{itemize}
      \item Исполнение запроса --- много одинаковых операций над кортежами.
      \item Часто приходится вычислять выражения.
      \item JIT-компиляция позволяет сгенерировать эффективный код для
            выражений.
    \end{itemize}
  \end{alertblock}
\end{frame}

\begin{frame}{Мотивация}
    SELECT \textbf{users.age > 18}, \textbf{groups.id}\\ FROM users\\ LEFT JOIN groups ON \textbf{users.group\_id =
    groups.id}\\ WHERE \textbf{users.age > 18};
\end{frame}

\begin{frame}{Цели и задачи}
  \metroset{block=fill}
  \begin{alertblock}{Цель}
    Реализация модельной реляционной СУБД с поддержкой опциональной
    JIT-компиляции выражений в SELECT запросах на основе LLVM-JIT
  \end{alertblock}
  \begin{alertblock}{Задачи}
    \begin{itemize}
      \item Изучить подходы к реализации СУБД и JIT-компиляции;
      \item Разработать архитектуру модельной СУБД;
      \item Реализовать модельную СУБД;
      \item Протестировать реализацию СУБД модульными тестами и измерить производительность.
    \end{itemize}
  \end{alertblock}
\end{frame}

\begin{frame}{Архитектура модельной СУБД}
  \begin{figure}[H]
    \centering
    \begin{minipage}[t]{\textwidth}
      \centering
      \makebox[\textwidth][c]{%
        \includegraphics[width=1.15\textwidth,height=0.75\textheight,keepaspectratio]{dbms_scheme_impl}%
      }
    \end{minipage}
    \caption{Схема верхнеуровнего устройства модельной СУБД.}\label{fig:dbms-scheme}
  \end{figure}
\end{frame}

\begin{frame}{Лексический и синтаксический анализ SQL}
  \begin{alertblock}{}
    \begin{itemize}
      \item ANTLR4
      \item Грамматика PostgreSQL из официального репозитория
    \end{itemize}
  \end{alertblock}
\end{frame}

\begin{frame}[fragile]{Лексический и синтаксический анализ SQL}
\begin{lstlisting}
(root
  (stmtblock
    (stmtmulti
      (stmt
        (selectstmt
          (select_no_parens
            (select_clause
              (simple_select_intersect
                (simple_select_pramary SELECT
                  (target_list_ (target_list (target_el *)))
                  (from_clause FROM (from_list
                      (table_ref
                        (relation_expr
                          (qualified_name
                            (colid
                              (identifier users))))))))))))) ;)) <EOF>)
\end{lstlisting}
\end{frame}

\begin{frame}{Преобразование в реляционную алгебру}
  \begin{center}
    SELECT users.id, groups.id FROM users LEFT JOIN groups ON users.group\_id =
    groups.id WHERE users.age > 18;
  \end{center}
    \begin{figure}[H]
      \centering
      \begin{minipage}[t]{\textwidth}
        \centering
        \includegraphics[width=0.5\textwidth]{query}
      \end{minipage}
      \caption{План запроса}\label{fig:query-plan}
    \end{figure}
\end{frame}

\begin{frame}{Преобразование в реляционную алгебру}
  \begin{alertblock}{}
    \begin{itemize}
      \item Кодогенерация Visitor
      \item Неподдерживаемый синтаксис = ошибка
      \item std::any
    \end{itemize}
  \end{alertblock}
\end{frame}

\begin{frame}{Исполнение запроса}
  \begin{alertblock}{}
    \begin{itemize}
      \item Векторизованная модель
      \item Коммуникация через каналы
      \item Данные текут снизу-вверх по плану исполнения
    \end{itemize}
  \end{alertblock}
\end{frame}

\begin{frame}{Реализация модельной СУБД}
  \begin{alertblock}{}
    \begin{itemize}
      \item C++
      \item Boost.Asio и C++20 корутины
      \item boost::asio::experimental::concurrent\_channel
      \item LLVM ORCv2 API
    \end{itemize}
  \end{alertblock}
\end{frame}

\begin{frame}{JIT-компиляция}
  Реализована компиляция выражений
  \begin{alertblock}{Оптимизационные проходы}
    \begin{itemize}
      \item EarlyCSEPass --- раннее устранение общих подвыражений;
      \item SROAPass --- скалярная замена агрегатов;
      \item InstCombinePass --- комбинирование инструкций, включая сворачивание констант;
      \item SimplifyCFGPass --- упрощение графа потока управления;
      \item ReassociatePass --- переассоциация выражений;
      \item GVNPass --- глобальная нумерация значений;
    \end{itemize}
  \end{alertblock}
\end{frame}

\begin{frame}{JIT-компиляция}
  Реализована компиляция выражений с кэшированием результатов
  \begin{alertblock}{Оптимизационные проходы}
    \begin{itemize}
      \item MemCpyOptPass --- оптимизация копирований памяти;
      \item SimplifyCFGPass --- повторное упрощение графа потока управления;
      \item InstCombinePass --- повторное комбинирование инструкций;
      \item ADCEPass --- агрессивное удаление мёртвого кода.
    \end{itemize}
  \end{alertblock}
\end{frame}

\begin{frame}{Тестирование}
  \begin{alertblock}{Для проверки корректности:}
  \begin{itemize}
    \item покрытие кода юнит-тестами
  \end{itemize}
  \end{alertblock}
\end{frame}

\begin{frame}{Бенчмарки}
  \begin{alertblock}{}
    \begin{itemize}
      \item google benchmarks
      \item сравнение jit-компиляции выражений и прямой интерпретации
    \end{itemize}
  \end{alertblock}
\end{frame}

\begin{frame}{Бенчмарки}
  \begin{figure}[H]
    \centering
    \begin{minipage}[t]{\textwidth}
      \centering
      \includegraphics[width=\textwidth]{perf}
    \end{minipage}
    \caption{Результаты бенчмарков.}\label{fig:perf}
  \end{figure}
\end{frame}

\begin{frame}{Бенчмарки}
  \begin{figure}[H]
    \centering
    \begin{minipage}[t]{\textwidth}
      \centering
      \includegraphics[width=\textwidth]{speedup}
    \end{minipage}
  \caption{Относительный прирост производительности.}\label{fig:speedup}
  \end{figure}
\end{frame}

\begin{frame}{Заключение}
  \begin{alertblock}{Возможные улучшения}
  \begin{itemize}
    \item Более полная поддержка синтаксиса SQL
    \item Реализация продвинутых алгоритмов исполнения операторов
    \item Бенчмарки без учета IO
  \end{itemize}
  \end{alertblock}
\end{frame}

\begin{frame}{Заключение}
  Цель курсовой работы была достигнута. Получена система, демонстрирующая
  возможности JIT-компиляции для оптимизации SELECT запросов.
\end{frame}

\begin{frame}{Заключение}
  В ходе разработки были получены следующие навыки:
    \begin{itemize}
      \item разработка многопоточных, конкурентных программ на C++;
      \item системное программирование на C++ в среде Linux;
      \item реализация парсера и исполнителя SQL запросов;
      \item реализация JIT-компиляции при помощи LLVM.
    \end{itemize}
\end{frame}

\end{document}
